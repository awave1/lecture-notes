\documentclass[11pt]{article}

% \usepackage[shortlabels]{enumerate}
\usepackage{algorithm, algorithmic, verbatim,amsthm,amsmath,amssymb,amsfonts,url,tcolorbox}

\usepackage[margin=1in]{geometry}

\parindent 0pt
\parskip 3mm

\theoremstyle{definition}
\newtheorem*{solution}{Solution}

\renewcommand{\pmod}[1]{\mbox{\ $(\ensuremath{\operatorname{mod}}\ {#1})$}}
\newcommand{\Z}{\mathbb{Z}}
\newcommand{\ord}{\mbox{$\ensuremath{\operatorname{ord}}$}}
\providecommand{\Leg}[2]{\genfrac{(}{)}{}{}{#1}{#2}}
\def\dsp#1{\displaystyle{#1}}

\newcommand{\A}{Alice}
\newcommand{\B}{Bob}
\newcommand{\E}{Eve}
\newcommand{\M}{Mallory}

\begin{document}

\begin{center}
  \bf \Large CPSC 418 / MATH 318 --- Introduction to Cryptography \\
  ASSIGNMENT 3
\end{center}

\medskip \hrule
  \textbf{Name:} Artem Golovin \\
  \textbf{Student ID:} 30018900
\medskip \hrule


\begin{enumerate} \itemsep 0.3in

\item[] \textbf{Problem 1} --- A modified man-in-the-middle attack on Diffie-Hellman, 12 marks

\begin{enumerate}
  \item
    Let $y_a \equiv (g^a)^q \pmod{p}$, $y_b = (g^b)^q \pmod{p}$ and key $K$:
    \begin{enumerate}
      \item $\A$ receives malicious $y_a$ and sends it to $\B$.
      \item $\B$ receives malicious $y_b$ and sends it to $\A$.
      \item $\A$ computes $K \equiv y_b^a \equiv ((g^b)^q)^a \pmod{p}$
      \item $\B$ computes $K \equiv y_a^b \equiv ((g^a)^q)^b \pmod{p}$
      \item $\A$ and $\B$ get the same key $K$, because:
        \begin{align*}
          y_b^a \equiv ((g^b)^q)^a \equiv g^{bqa} \equiv g^{aqb} \equiv ((g^a)^q)^b \equiv y_b^a \pmod{p}
        \end{align*}
    \end{enumerate}
  \item ???
  \item 
    In this version, $\M$ does not have to pick a number $e$, where $1 < e < p$. Therefore, by knowing values $g^a \pmod{p}$ and $g^b \pmod{p}$, $\M$ is more likely to compute $g^{abq} \pmod{p}$, which is a private key used by $\A$ and $\B$.
\end{enumerate}

\newpage

\item[] \textbf{Problem 2} --- RSA and binary exponentiation, 24 marks

\begin{enumerate}
  \item ok
\end{enumerate}

\newpage

\item[] \textbf{Problem 3} --- 

\begin{enumerate}
  \item ok
\end{enumerate}

\newpage

\item[] \textbf{Problem 4} --- The ElGamal public key cryptosystem is not semantically secure, 10 marks

\begin{proof}
  By definition, a PKC is polynomially secure if no passive attacker can in expected polynomial time select two plaintexts $M_1$ and $M_2$ and then correctly distinguish between $E(M_1)$ and $E(M_2)$, where $E(M_1)$ and $E(M_2)$ are encryptions of $M_1$ and $M_2$ respectively with probability $p > \frac{1}{2}$.

  However, it is given that $\M$ can assert whether $C = E(M_1)$ or $C = E(M_2)$ in polynomial time using modular exponentiation by Euler's Criterion with probability $p' = 1$, $p' > p$. It contradicts the definition of polynomially secure PKC, and therefore shows that ElGamal is not semantically secure.
\end{proof}

\newpage

\item[] \textbf{Problem 5} --- 

\begin{enumerate}
  \item ok
\end{enumerate}

\newpage

\end{enumerate}
\end{document}