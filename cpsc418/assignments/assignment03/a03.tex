\documentclass[11pt]{article}

% \usepackage[shortlabels]{enumerate}
\usepackage{algorithm, algorithmic, verbatim,amsthm,amsmath,amssymb,amsfonts,url,tcolorbox}

\usepackage[margin=1in]{geometry}

\parindent 0pt
\parskip 3mm

\theoremstyle{definition}
\newtheorem*{solution}{Solution}

\renewcommand{\pmod}[1]{\mbox{\ $(\ensuremath{\operatorname{mod}}\ {#1})$}}
\newcommand{\Z}{\mathbb{Z}}
\newcommand{\ord}{\mbox{$\ensuremath{\operatorname{ord}}$}}
\providecommand{\Leg}[2]{\genfrac{(}{)}{}{}{#1}{#2}}
\def\dsp#1{\displaystyle{#1}}
\newcommand{\xor}{\oplus}

\newcommand{\A}{Alice}
\newcommand{\B}{Bob}
\newcommand{\E}{Eve}
\newcommand{\M}{Mallory}

\begin{document}

\begin{center}
  \bf \Large CPSC 418 / MATH 318 --- Introduction to Cryptography \\
  ASSIGNMENT 3
\end{center}

\medskip \hrule
  \textbf{Name:} Artem Golovin \\
  \textbf{Student ID:} 30018900
\medskip \hrule


\begin{enumerate} \itemsep 0.3in

\item[] \textbf{Problem 1} --- A modified man-in-the-middle attack on Diffie-Hellman, 12 marks

\begin{enumerate}
  \item
    Let $y_a \equiv (g^a)^q \pmod{p}$, $y_b = (g^b)^q \pmod{p}$ and key $K$:
    \begin{enumerate}
      \item $\A$ receives malicious $y_a$ and sends it to $\B$.
      \item $\B$ receives malicious $y_b$ and sends it to $\A$.
      \item $\A$ computes $K \equiv y_b^a \equiv ((g^b)^q)^a \equiv g^{abq} \pmod{p}$
      \item $\B$ computes $K \equiv y_a^b \equiv ((g^a)^q)^b \equiv g^{abq} \pmod{p}$
      \item $\A$ and $\B$ get the same key $K$, because:
        \begin{align*}
          y_b^a \equiv ((g^b)^q)^a \equiv g^{bqa} \equiv g^{aqb} \equiv ((g^a)^q)^b \equiv y_b^a \pmod{p}
        \end{align*}
    \end{enumerate}
  \item
    Given that $g$ is a primitive root of $p$ and $p - 1 = mq$, then $g^{(p - 1)} \equiv g^{mq} \equiv 1 \pmod{p}$, by Fermat's Little Theorem. We know that $K \equiv g^{abq} \pmod{p}$

    Let $k \in \Z$ and $k \geq 1$,
    \begin{align*}
      g^{kmq} \equiv (g^{mq})^k \equiv (g^{(p - 1)})^k \equiv 1 \pmod{p}
    \end{align*}

    Suppose $ab$ = $km$, then we'll get:
    \begin{align*}
      g^{abq} \equiv g^{kmq} \equiv 1 \pmod {p}
    \end{align*}

    Therefore, there are $m$ possible values for $K$.
  \item
    In this version, $\M$ sends $g^{aq}$ or $g^{bq}$, and since $\M$ knows $g^{a} \pmod{p}$ and $g^{b} \pmod{p}$, she is able to compute $g^{abq} \pmod{p}$ which is the same key that $\A$ and $\B$ use. In the example, discussed in class $\E$  has no knowledge of shared key that $\A$ and $\B$ use and therefore $\E$ would need to handle decryption and encryption separatly.
\end{enumerate}

\newpage

\item[] \textbf{Problem 2} --- RSA and binary exponentiation, 24 marks

\begin{enumerate}
  \item
    \begin{enumerate}
      \item 
        Given $M = 17$, public key $(e, n) = (11, 77)$:
        \begin{align*}
          C \equiv M^e \pmod{n} \\
          C \equiv 17^{11} \pmod{77}
        \end{align*}

        Binary exponentiation:
        \begin{align*}
          e = 11 = 1011_{2} \\
          b_0 = 1, b_1 = 0, b_2 = 1, b_3 = 1 \\
          r_0 &\equiv 17 \pmod{77} \\
          r_1 &\equiv 17^2 \equiv 58 \pmod{77} \\
          r_2 &\equiv 58^2 \times 17 \equiv 54 \pmod{77}  \\
          r_3 &\equiv 54^2 \times 17 \pmod{77} \equiv 61 \pmod{77}  \\
        \end{align*}

        Therefore:
        \begin{align*}
          C \equiv 17^{11} \equiv 61 \pmod{77}
        \end{align*}
      \item
        Given that $n = pq$ and $\phi{(n)} = (p - 1)(q - 1)$ and $n = 77$, we can say that $p = 11$, $q = 7$.

        To find $d$, we need to solve the following congruence:
        \begin{align*}
          de &\equiv 1 \pmod{\phi{(n)}} \\
             &\text{Where } \phi{(n)} = (11 - 1)(7 - 1) = 60 \\
          d \times 11 &\equiv 1 \pmod{60}
        \end{align*}

        Solving $gcd(e, \phi{(n)}) = gcd(11, 60) = 1$ to confirm that inverse of $e$ exists:
        \begin{align*}
          60 &= 11 \times 5 + 5 \\
          11 &= 10 \times 1 + 1 \\
          10 &= 2 \times 5 + 0 \\
        \end{align*}

        Applying Extended Euclidean Algorithm to find $d$:
        \begin{align*}
          1 &= 11 - 10 = 11 - ((2 \times 5) + 0) = 11 - (2 \times 5) - 0 \\
            &= 11 - 2 \times (60 - 11 \times 5) = 11 - 2 \times 60 + 10 \times 11 \\
            &= 11 \times 11 + (-2) \times 60
        \end{align*}
        Therefore $gcd(60, 11) = 11 \times 11 + (-2) \times 60$ and $d = 11$:
        \begin{align*}
          11 \times 11 \equiv 1 \pmod{60}
        \end{align*}
      \item
        Given $C = 21$ and $(d, n) = (11, 77)$:
        \begin{align*}
          M &\equiv C^d \pmod{n} \\
          M &\equiv 32^{11} \pmod{77}
        \end{align*}

        Binary exponentiation:
        \begin{align*}
          11 = 1011_{2} \\
          b_0 = 1, b_1 = 0, b_2 = 1, b_3 = 1 \\
          r_0 &\equiv 32 \pmod{77} \\
          r_1 &\equiv 32^2 \equiv 23 \pmod{77} \\
          r_2 &\equiv 23^2 \times 32 \equiv 16928 \equiv 65 \pmod{77} \\
          r_3 &\equiv 65^2 \times 32 \equiv 135200 \equiv 65 \pmod{77}
        \end{align*}
        Therefore, $M = 65$
    \end{enumerate}
  \item
    \begin{enumerate}
      \item
        Given $s_0 = b_0$, $s_{i + 1} = 2s_i + b_{i + 1}$ for $0 \leq i \leq k - 1$
        \begin{proof}
          Base case. Let $i = 0$
          \begin{align*}
            s_i &= \sum^{i}_{j = 0} b_j 2^{i - j} \\
            s_0 &= \sum^{0}_{j = 0} b_j2^{0 - 0} \\
                &= b_0
          \end{align*}

          Induction hypothesis. Assume $i = m$, where $m \in \Z$ and $0 \leq m \leq k$:
          \begin{align*}
            s_m &= \sum^{m}_{j = 0} b_j2^{m - j} \\
                &= b_0 + \sum^{m + 1}_{j = 0} b_j 2^{(m + 1) - j}
          \end{align*}

          Inductive case. Assume $i = m + 1$:
          \begin{align*}
            s_{m + 1} = \sum^{m + 1}_{j = 0} b_j 2^{(m + 1) - j} \\
            \text{Left hand side: } \\
            s_{m + 1} = 2s_m + b_{m + 1} \\
            \text{Right hand side: } \\
            \sum^{m + 1}_{j = 0} b_j 2^{(m + 1) - j} &= 2 \times \sum^{m}_{j = 0} b_j 2^{(m - j)} + b_{m + 1} \\
                                                     &= 2s_m + b_{m + 1} \\ \\
            \text{Therefore, } s_i &= \sum^{i}_{j = 0} b_j 2^{i - j}
          \end{align*}
        \end{proof}
      \item
        Let $r_i$, $0 \leq i \leq k$, $k = \left \lfloor{log_2{n}}\right \rfloor$.
        \begin{proof}
          Base case. Let $i = 0$:
          \begin{align*}
            r_i &\equiv a^{s_i} \pmod{m} \\
            r_0 &\equiv a^{s_0} \pmod{m} \\
                &\equiv a^{b_0} \equiv a \pmod{m}
          \end{align*}
          Where $s_0 = b_0$ is shown in part $(i)$

          Induction hypothesis. Let $p = i$, $p \in \Z$, $0 \leq p \leq k$,
          \begin{align*}
            r_p \equiv a^{s_p} \pmod{m}
          \end{align*}

          Inductive case. Show $r_{p + 1} \equiv a^{s_{p + 1}} \pmod{m}$.
          \begin{align*}
            \text{\textbf{Case }} b_{i + 1} = 0 \\
            r_{p + 1} \equiv r^2_p \pmod{m} \\
            \text{Therefore:} \\
            a^{s_p + 1} &\equiv a^{2s_p + b_{p + 1}} \\
                        &\equiv a^{2s_p} \\
                        &\equiv a^{s_p} \times a^{sp} \\
                        &\equiv r_{p} \times r_{p} \\
                        &\equiv r^2_{p} \pmod{m}
          \end{align*}

          \begin{align*}
            \text{\textbf{Case }} b_{i + 1} = 1 \\
            a^{s_p + 1} &\equiv a^{2s_p + b_{p + 1}} \\
                        &\equiv a^{s_p} \times a^{s_p} \times a^{b_{p + 1}} \\
                        &\equiv r^2_{p} \times a^{b_{p + 1}} \\
                        &\equiv r^2_{p} \times a \pmod{m}
          \end{align*}

          Therefore, $r_{p + 1} \equiv a^{p + 1} \pmod{m}$ and $r_i \equiv a^i \pmod{m}$
        \end{proof}
      \item
        \begin{proof}
          Given proof of $(ii)$, we can say that $a^n \equiv r_k \pmod{m}$, where $n = s_k$. Therefore,
          \begin{align*}
            a^{s_k} &\equiv a^{2s_{k - 1} + b_k} \\
                    &\equiv (a^{s_{k - 1}})^2 \times a^{b_k} \\
                    &\equiv (r_{k - 1})^2 \times a^{b_k} \\
                    &\equiv r_k \pmod{m}
          \end{align*}
        \end{proof}
    \end{enumerate}
\end{enumerate}

\newpage

\item[] \textbf{Problem 3} --- Fast RSA decryption using Chinese remaindering, 8 marks
 
Let $p$ and $q$ be distinct large primes and $(e, n)$ be RSA public key with corresponding private key $d$. Given $d_p \equiv d \pmod{p - 1}$ and $d_q \equiv d \pmod{q - 1}$
\begin{align*}
  M_p &\equiv C^{dp} \equiv C^{d}(C^{p - 1})^{k} \\
      &\equiv C^{d} \\
      &\equiv M^{ed} \pmod{p} \\
  M_q &\equiv C^{dq} \equiv C^{d}(C^{q - 1})^{k} \\
      &\equiv C^{d} \\
      &\equiv M^{ed} \pmod{q}
\end{align*}

Therefore,
\begin{align*}
  M_p &\equiv M^{ed} \pmod{p} \\
  M_q &\equiv M^{ed} \pmod{q} \\
  M_p &= M^{ed} + qt \\
  M_q &= M^{ed} + pk
\end{align*}

For some $t \in \Z$ and $k \in \Z$.

Let $M' \equiv pxM_q + qyM_p \pmod{n}$ be the plaintext obtained using Chinese remainer theorem and $M \equiv C^d \pmod{n}$ be the message obtained using normal RSA way.
\begin{align*}
  M \equiv C^{d} \equiv (M^e)^d \equiv M^{ed} \\
  \text{Where } ed \equiv 1 \pmod{\phi{(n)}} \\
  \text{Therefore } ed = 1 + m\phi{(n)} \text{ for some } m \in \Z \\
  M \equiv M^{1 + m\phi{(n)}} \equiv MM^{m\phi{(n)}} \equiv M(M^{\phi{(n)}})^m \pmod{n} \\ \\
  \text{by Euler's theorem: } \\
  M^{\phi{(n)}} \equiv 1 \pmod{n} \\
  \text{which gives us: } \\
  M \equiv M(M^{\phi{(n)}})^m \equiv M \pmod{n}
\end{align*}

Plaintext $M'$, obtained using Chinese remainer theorem:
\begin{align*}
  M' &\equiv pxM_q + qyM_p \\
     &\equiv px(M^{ed} + qt) + qy(M^{ed} + pk) \\
     &\equiv pxM^{ed} + pxqt + qyM^{ed} + qypk \\
     &\equiv pxM^{ed} + qyM^{ed} \\
     &\equiv M^{ed}(px + qy) \\
     &\text{where } (px + qy) = 1 \text{ because gcd(\textit{p}, \textit{q}) = 1} \\
     &\equiv M^{ed} \pmod{n} \\
     &\equiv M \pmod{n}
\end{align*}

Therefore $M = M'$ \qed

\newpage

\item[] \textbf{Problem 4} --- The ElGamal public key cryptosystem is not semantically secure, 10 marks

By definition, a PKC is polynomially secure if no passive attacker can in expected polynomial time select two plaintexts $M_1$ and $M_2$ and then correctly distinguish between $E(M_1)$ and $E(M_2)$, where $E(M_1)$ and $E(M_2)$ are encryptions of $M_1$ and $M_2$ respectively with probability $p > \frac{1}{2}$.

  However, it is given that $\M$ can assert whether $C = E(M_1)$ or $C = E(M_2)$ in polynomial time using modular exponentiation by Euler's Criterion with probability $p' = 1$, $p' > p$. It contradicts the definition of polynomially secure PKC, and therefore shows that ElGamal is not semantically secure.

  Let $C_1 \equiv g^k \pmod{p}$, $C_2 \equiv My^{k} \pmod{p}$, $C^{p - 1 - x}_1y^{k} \equiv M \pmod{p}$ and $y \equiv g^x \pmod{p}$

  \begin{enumerate}
    \item[] \textbf{Case} $\Leg{y}{p} = 1$
      If $\Leg{y}{p} = 1$, then 
      \begin{enumerate}
        \item[] Calculating $\Leg{C_2}{p}$
          \begin{align*}
            \Leg{C_2}{p} &= \Leg{My^k}{p} = \Leg{Mg^{xk}}{p}
          \end{align*}
          Therefore, $Mg^{xk} \in QR_p$ if and only if $\Leg{Mg^{xk}}{p} = 1$. By Euler's Criterion, we get $(Mg^{xk})^{\frac{p - 1}{2}} \equiv \Leg{Mg^{xk}}{p} \equiv 1 \pmod{p}$
          \begin{align*}
            (Mg^{xk})^{\frac{p - 1}{2}} &\equiv M^{\frac{p - 1}{2}}(g^{p - 1})^{\frac{xk}{2}} \\
                                        &\equiv M^{\frac{p - 1}{2}} \pmod{p}
          \end{align*}

      \end{enumerate}
    \item[] \textbf{Case} $\Leg{y}{p} = -1$
      \begin{enumerate}
        \item[] Calculating $\Leg{C_1}{p}$
          \begin{align*}
            \Leg{C_1}{p} &= \Leg{g^{k}}{p} \\
                         &= \Leg{g^{k} \pmod{p}}{p}
          \end{align*}
          If $\Leg{C_1}{p} = 1$, then $C_1 \in QR_p$ and $C_1^{\frac{(p - 1)}{2}} \equiv 1 \pmod{p}$ according to Euler's Criterion
          \begin{align*}
          (g^k)^{\frac{p - 1}{2}} &\equiv (g^{k(p - 1)})^{\frac{1}{2}} \\
                                  &\equiv (g^{(p - 1)})^{\frac{k}{2}} \\
                                  &\equiv 1 \pmod{p} \text{ (by Fermat's Little Theorem)}
          \end{align*}

        \item[] Calculating $\Leg{C_2}{p}$
          \begin{align*}
            \Leg{C_2}{p} &= \Leg{My^k}{p} = \Leg{Mg^{xk}}{p}
          \end{align*}
          Therefore, $Mg^{xk} \in QR_p$ if and only if $\Leg{Mg^{xk}}{p} = 1$. By Euler's Criterion, we get $(Mg^{xk})^{\frac{p - 1}{2}} \equiv \Leg{Mg^{xk}}{p} \equiv 1 \pmod{p}$
          \begin{align*}
            (Mg^{xk})^{\frac{p - 1}{2}} &\equiv M^{\frac{p - 1}{2}}(g^{p - 1})^{\frac{xk}{2}} \\
                                        &\equiv M^{\frac{p - 1}{2}} \pmod{p}
          \end{align*}

      \end{enumerate}
  \end{enumerate}

\newpage

\item[] \textbf{Problem 5} --- An IND-CPA, but not IND-CCA secure version of RSA, 10 marks 

\begin{proof}
  Given encryption of message $M$, $C = (s || t)$, where $s \equiv r^e \pmod{n}$, $t = H(r) \xor M$ and $H: \{0, 1\}^k \mapsto \{0, 1\}^m$, and decryption of $C$, $M \equiv H(s^d \pmod{n}) \xor t$, we can consider two plaintexts $M_1$ and $M_2$ with following enryption process: $C = (s||t) = (r^e (\pmod{n}) || H(r) \xor M_i)$, where $i = 1$ or $2$.

  We can mount CCA using $C' = (s || t \xor M_1)$:
  \begin{align*}
    C' &= (s || t \xor M_1) \\
       &= (r^e (\pmod{n}) || H(r) \xor M_i \xor M_1)
  \end{align*}

  Decryption of $M_i$:
  \begin{align*}
    M_i &\equiv H(s^d (\pmod{n})) \xor t \\
        &\equiv H(r^{ed} (\pmod{n})) \xor H(r) \xor M_i \xor M_1 \\
        &\equiv H(r \pmod{n}) \xor H(r) \xor M_i \xor M_1 \\
        &= M_i \xor M_1
  \end{align*}
  Where $ed \equiv 1 \pmod{\phi(n)}$

  Therefore, $M_i = 0$, $C$ is an encryption of $M_1$, because $M_1 \xor M_1 = 0$, otherwise $M_i = M_2$
\end{proof}

\end{enumerate}
\end{document}
