\documentclass[11pt]{article}

\usepackage{fullpage, verbatim, amsthm, amsmath, amssymb, amsfonts}

% Aliases
\newcommand{\K}{\mathcal{K}}
\newcommand{\M}{\mathcal{M}}
\newcommand{\C}{\mathcal{C}}
\newcommand{\Z}{\mathbb{Z}}

% Setup page style
\parindent 0pt
\parskip 3mm

\theoremstyle{definition}
\newtheorem*{solution}{Solution}


\begin{document}

\begin{center}

  \bf \Large CPSC 418 / MATH 318 --- Introduction to Cryptography \\
  ASSIGNMENT 1

\end{center}


\medskip \hrule

  \textbf{Name:} Artem Golovin \\
  \textbf{Student ID:} 30018900

\medskip \hrule

% Problem 1
\item[] \textbf{Problem 1} --- Superencipherment for substitution ciphers, 12 marks

\begin{enumerate}
  \item
    \begin{enumerate}
      \item
        \begin{proof}
          Encryption using Shift cipher is given by $E_K (M) &\equiv (M + K) \mod 26$. Given $\M = \C = \K = \Z_{26}$, $K_1, K_2 \in \K$ and $M \in \M$: \\
          Let $C_1 \in \C$ be a ciphertext that results from encrypting plaintext $M$ with a key $K_1$:
          \begin{enumerate}
            \item $ E_{K_1}(M) &\equiv (M + K_1) \mod 26 $
            \item $ C_1 = E_{K_1} $
            \item $ C_1 = (M + K_1) \mod 26 $
            \item Let $C_2 = E_{K_2}(C_1)$, where $E_{K_2}(C_1) &\equiv (C_1 + K_2) \mod 26$
            \item Therefore, by substituting $C_1$,
              \begin{equation}
              \begin{aligned}
                C_2 & = (C_1 + K_2) \mod 26 \\
                    & = ((M + K_1) + K_2) \mod 26 \\
                    & = (M + K_3) \mod 26
              \end{aligned}
              \end{equation}
              Where $K_3 \in \K$ and $K_3 = K_1 + K_2$. Finally, according to definition, $C_2$ results in shift cipher.
          \end{enumerate}
        \end{proof}
      \item
    \end{enumerate}

  \item
\end{enumerate}

\newpage


\end{document}
