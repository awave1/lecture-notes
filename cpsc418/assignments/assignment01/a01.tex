\documentclass[11pt]{article}

\usepackage{fullpage, verbatim, amsthm, amsmath, amssymb, amsfonts}

% Aliases
\newcommand{\K}{\mathcal{K}}
\newcommand{\M}{\mathcal{M}}
\newcommand{\C}{\mathcal{C}}
\newcommand{\Z}{\mathbb{Z}}

% Setup page style
\parindent=0pt
\parskip=3mm

\theoremstyle{definition}
\newtheorem*{solution}{Solution}


\begin{document}

\begin{center}

  \bf \Large CPSC 418 / MATH 318 --- Introduction to Cryptography \\
  ASSIGNMENT 1

\end{center}


\medskip \hrule

  \textbf{Name:} Artem Golovin \\
  \textbf{Student ID:} 30018900

\medskip \hrule

% Problem 1
\item[] \textbf{Problem 1} --- Superencipherment for substitution ciphers, 12 marks

\begin{enumerate}
  \item
    \begin{enumerate}
      \item
        \begin{proof}
          Encryption using Shift cipher is given by $E_K (M) &\equiv (M + K) \mod 26$. Given $\M = \C = \K = \Z_{26}$, $K_1, K_2 \in \K$ and $M \in \M$: \\
          Let $C_1 \in \C$ be a ciphertext that results from encrypting plaintext $M$ with a key $K_1$:
          \begin{enumerate}
            \item $ E_{K_1}(M) &\equiv (M + K_1) \mod 26 $
            \item $ C_1 &\equiv E_{K_1} $
            \item $ C_1 &\equiv (M + K_1) \mod 26 $
            \item Let $C_2 = E_{K_2}(C_1)$, where $E_{K_2}(C_1) &\equiv (C_1 + K_2) \mod 26$
            \item Therefore, by substituting $C_1$,
              \begin{equation*}
              \begin{aligned}
                C_2 & \equiv (C_1 + K_2) \mod 26 \\
                    & \equiv ((M + K_1) + K_2) \mod 26 \\
                    & \equiv (M + K_3) \mod 26
              \end{aligned}
              \end{equation*}
              Where $K_3 \in \K$ and $K_3 = K_1 + K_2$. Therefore, resulting key of multiple encipherment is $K_3$.
          \end{enumerate}
        \end{proof}

      \item
        \begin{proof}
          Based on previous proof, superencipherment using shift cipher can be defined as follows
          \[
            E_{K_i}(M) &\equiv (M + \sum_{\substack{K_i \in \K \\ i = 1}}^n K_i) \mod 26
          \]
          Where $M \in \M$, $K_i \in \K$, $n \in \Z$, and $i \geq 1, i \in \Z$.
          \begin{enumerate}
            \item
              Base Case: let $n = 1$,
              \begin{equation} \label{eq:p1_base_case}
              \begin{aligned}
                C &\equiv (M + \sum_{\substack{K_i \in \K \\ i = 1}}^1 K_i) \mod 26 \\
                  &\equiv (M + K_1) \mod 26
              \end{aligned}
              \end{equation}
            \item
              Induction hypothesis: Assume $n = m$, where $m \in \Z$. Therefore:
              \[
                C &\equiv (M + \sum_{\substack{K_i \in \K \\ i = 1}}^m K_i) \mod 26
              \]

              Let: \\
              \begin{equation} \label{eq:p1_induction_hyp}
              \begin{aligned}
                K^{'} &= \sum^m_{\substack{K_i \in \K \\ i = 1}} K_i \\
                C &\equiv (M + K^{'}) \mod 26
              \end{aligned}
              \end{equation}
              Still results in a shift cipher, according to definition.
            \item
              Inductive case: According to induction hypothesis, we can show that $m + 1$ holds true as well:
              \begin{align*}
                C &\equiv (M + \sum^{m + 1}_{\substack{K_i \in \K \\ i = 1}} K_i) \mod 26 \\
                C &\equiv (M + (\sum^{m}_{\substack{K_i \in \K \\ i = 1}} K_i + \sum^{1}_{\substack{K_i \in \K \\ i = 1}} K_i)) \mod 26 \\
                C &\equiv (M + (\sum^{m}_{\substack{K_i \in \K \\ i = 1}} K_i + K_1)) \mod 26 \\
                C &\equiv (M + (K^{'} + K_1)) \mod 26 \\
                C &\equiv (M + K^{"}) \mod 26 \\
              \end{align*}
              where $K_1$ is our base case \eqref{eq:p1_base_case} and $K^{'}$ is our induction hypothesis \eqref{eq:p1_induction_hyp}. Since $K_1 \in \K$ and $K^{'} \in \K$, therefore $K^{"} = K_1 + K^{'}, K^{"} \in \K$. Hence, the key of multiple encipherment is sum of all given keys.
          \end{enumerate}
        \end{proof}
    \end{enumerate}

  \item
    \begin{proof}
      Let $M \in \M$ be a plaintext of length $x \in \Z$. Let $p_0, p_1, p_2, ...,p_{x - 1}$ be positions of letters in plaintext $M$. Given key $w_1 \in \K$ of length $m \in \Z$ and key $w_2 \in \K$ of length $n \in \Z$, let $k_0, k_1, k_2, ...,k_{m - 1}$ be positions of letters in key $w_1$, and $l_0, l_1, l_2, ...,l_{n - 1}$ be positions of letters in key $w_2$. To encrypt plaintext $p_i$, we use a key $k_j$, where $i$ is letter position from $0$ to $x - 1$.
      \[
        j \equiv i \pmod m
      \]
      Let ciphertext $C_i$ be ciphertext that corresponds to $p_i$
      \[
        C_i \equiv p_i + k_j \pmod {26}
      \]
      Where $k_j \equiv i \pmod m$. \\
      Let ciphertext $C_{2_i}$ be ciphertext that corresponds to $C_i$. Therefore, the second round of encryption, using the key $w_2$, results in following:
      \begin{equation*}
      \begin{aligned}
        C_{2_i} &\equiv C_i + l_j &\pmod {26} \\
                &\equiv (p_i + k_j) + l_z &\pmod {26} \\
                &\equiv p_i + (k_j + l_z) &\pmod {26}
      \end{aligned}
      \end{equation*}
      Where $z \equiv i \pmod n$ and $l_z \equiv i \pmod n$. Therefore that ensures that length of resulting key will be $x$.
    \end{proof}
\end{enumerate}

\newpage

\item[] \textbf{Problem 2} --- Key size versus password size, 21 marks



\newpage

\item[] \textbf{Problem 3} --- Equiprobability maximizes entropy for two outcomes, 12 marks

\end{document}
